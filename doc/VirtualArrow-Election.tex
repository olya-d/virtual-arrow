\haddockmoduleheading{VirtualArrow.Election}
\label{module:VirtualArrow.Election}
\haddockbeginheader
{\haddockverb\begin{verbatim}
module VirtualArrow.Election (
    bordaCount,  oneDistrictProportionality,  plurality,  runOffPlurality, 
    multiDistrictProportionality,  mixedMember1,  mixedMember2, 
    thresholdProportionality,  singleTransferableVote
  ) where\end{verbatim}}
\haddockendheader

\begin{haddockdesc}
\item[\begin{tabular}{@{}l}
bordaCount\ ::\ Input\ ->\ Parliament
\end{tabular}]\haddockbegindoc
Find the Borda Winner in each district. The winner gets all seats.
 | Sum seats across districts.\par

\end{haddockdesc}
\begin{haddockdesc}
\item[\begin{tabular}{@{}l}
oneDistrictProportionality\ ::\ Input\ ->\ Parliament
\end{tabular}]\haddockbegindoc
The districts are aggregated, the seats in the parliament are distributed
 | according to the shares of (first) votes in the electorate.\par

\end{haddockdesc}
\begin{haddockdesc}
\item[\begin{tabular}{@{}l}
plurality\ ::\ Input\ ->\ Parliament
\end{tabular}]\haddockbegindoc
The candidate (i.e. the party) with the most votes wins in each district.\par

\end{haddockdesc}
\begin{haddockdesc}
\item[\begin{tabular}{@{}l}
runOffPlurality\ ::\ Input\ ->\ Parliament
\end{tabular}]\haddockbegindoc
In each district all parties but the two with the most votes are excluded.
 | The second round is implemented with these two parties only and 
 | the one with the most votes wins. 
 | If after the first round the first party has at least 50{\char '45} of the votes, 
 | it is elected without the need of a second round. \par

\end{haddockdesc}
\begin{haddockdesc}
\item[
multiDistrictProportionality\ ::\ Input\ ->\ Parliament
]
\end{haddockdesc}
\begin{haddockdesc}
\item[\begin{tabular}{@{}l}
mixedMember1\ ::\ Input\ ->\ Double\ ->\ Parliament
\end{tabular}]\haddockbegindoc
One parliament is elected with the Plurality System,
 | and one with Proportional System.
 | The resulting parliament is a weighted mean of the two.
 | Weight of the first parliament relative to the second is specified.\par

\end{haddockdesc}
\begin{haddockdesc}
\item[\begin{tabular}{@{}l}
mixedMember2\ ::\ Input\ ->\ Double\ ->\ Parliament
\end{tabular}]\haddockbegindoc
Part of parliament is elected with the Plurality System is computed,
 | and the remainder is elected using the Proportional System.
 | Share of seats to be elected with the Plurality System is specified.\par

\end{haddockdesc}
\begin{haddockdesc}
\item[\begin{tabular}{@{}l}
thresholdProportionality\ ::\ Input\ ->\ Double\ ->\ Parliament
\end{tabular}]\haddockbegindoc
All the parties who have a share of votes (strictly) smaller  
 | are excluded from the parliament. 
 | The seats are distributed proportionally among the remaining parties. 
 | There is only one district.\par

\end{haddockdesc}
\begin{haddockdesc}
\item[\begin{tabular}{@{}l}
singleTransferableVote\ ::\ Input\ ->\ Map\ Int\ Int\ ->\ Parliament
\end{tabular}]\haddockbegindoc
Single Transferable Vote is used with the list of candidates,
 | thus the additional parameter (map) is required to determine
 | the number of seats for each party.\par

\end{haddockdesc}